\documentclass[a4j, titlepage, 11pt]{jsarticle}

\usepackage{fancybox}
\usepackage{ascmac}
\usepackage{amsmath}
\usepackage[dvipdfm]{graphicx}
\usepackage{subfigure}
\usepackage{enumerate}
\usepackage{multicol}
\usepackage{float}

\setlength\abovecaptionskip{5pt} 
\setlength\belowcaptionskip{5pt}
\setlength\textfloatsep{5pt}
\setlength\intextsep{5pt}

\renewcommand{\labelenumi}{(\arabic{enumi})}

\setcounter{tocdepth}{3}

\def\vector#1{\mbox{\boldmath $#1$}}

%% 高さの設定 %%
\setlength{\textheight}{\paperheight}   % ひとまず紙面を本文領域に
\setlength{\topmargin}{-24truemm}      % 上の余白を20mm(=1inch-0.4mm)に
\addtolength{\topmargin}{-\headheight}  % 
\addtolength{\topmargin}{-\headsep}     % ヘッダの分だけ本文領域を移動させる
\addtolength{\textheight}{-24truemm}    % 下の余白も20mmに

%%  幅の設定 %%
\setlength{\textwidth}{\paperwidth}     % 紙面横幅を本文領域にする(RIGHT=-LEFT)
\setlength{\oddsidemargin}{-20truemm}  % 左の余白を25mm(=1inch-0.4mm)に
\setlength{\evensidemargin}{-0.4truemm} % 
\addtolength{\textwidth}{-20truemm}     % 右の余白も25mm(RIGHT=-LEFT)

%% プリアンプル部 %%

\begin{document}

\title{創造情報学専攻 専門科目 問4対策}
\author{@SythonUK}
\date{\today}

\maketitle

\tableofcontents
\clearpage

\section{既出問題}
\begin{multicols}{2}
\begin{itemize}
	\item スーパースカラ
	\item 半加算器
	\item キャリー・ルック・アヘッド
	\item TLB
	\item パイプラインハザード
	\item マイクロプログラム制御
	\item スヌープキャッシュ
	\item アウトオブオーダ実行
	\item RISCとCISC
	\item プロセスとスレッド
	\item クロスサイトスクリプティング
	\item 公開鍵暗号
	\item PKI
	\item ユーザ認証 (個人識別) 法
	\item フィッシング
	\item 公開鍵のディジタル署名
	\item PWM制御
	\item ステップ応答と伝達関数
	\item カルマンフィルタ
	\item フィードバック/フォワード制御
	\item $\mathrm{H}^{\infty}$制御
	\item サーボ系の位置とトルクフィードバック
	\item インパルス応答とステップ応答
	\item PID制御
	\item 交流回路のインピーダンス
	\item 実行時コンパイラ
	\item クロージャ
	\item 有限オートマトン
	\item 末尾再帰
	\item LL(1)構文解析
	\item 高階関数
	\item 関係データベースの正規化
	\item チューリングマシン
	\item リフレクション
	\item 参照透明性
	\item 正規文法と正規言語
	\item 関数型言語
	\item 自然言語の形態素
	\item 関係データベースの結合演算
	\item 文脈自由文法
	\item 論理型言語
	\item サポートベクターマシン
	\item tf-idf
	\item ベイジアンネットワーク
	\item ニューラルネットの学習
	\item 主成分分析
	\item ベイズの定理
	\item 決定木の学習
	\item 隠れマルコフモデル
	\item ベクトル量子化
	\item 意味ネットワーク
	\item 最尤推定法
	\item 中心極限定理
	\item ひずみゲージ
	\item 不確かさ
	\item 近接覚センサの原理
	\item 同時座標系
	\item グローシェーディング
	\item フォンシェーディング
	\item モンテカルロ法
	\item 分割統治法
	\item B木
	\item 分枝限定法
	\item ヒープソート
	\item 分散ハッシュ
	\item DNS
	\item クライアントサーバシステム
	\item P2Pシステム
	\item AS番号
	\item スペクトル拡散通信
	\item Unicode
	\item グリッドコンピューティング
	\item CGI
	\item TCPとUDP
	\item オプティカルフロー
	\item 離散コサイン変換
	\item 画像のエッジ検出法
	\item ナイキスト周波数
	\item サンプリング定理
	\item GIFとJPEGの符号化
	\item ハフ変換
	\item ラプラス変換
\end{itemize}
\end{multicols}

\section{プログラミング}
\begin{multicols}{2}
\begin{itemize}
	\item プリプロセッサ
	\item コンパイラ
	\item アセンブラ
	\item アセンブリ言語
	\item ニーモニック
	\item リンカ (リンケージエディタ)
	\item インタプリタ
	\item デバッガ
	\item オブジェクト指向
	\item クラス変数とインスタンス変数
	\item 継承と継承
	\item オーバーライドとオーバーロード
	\item ポリモーフィズム
	\item イミュータブル
	\item ジェネリックプログラミング
	\item 命令型(手続き型)プログラミング
	\item 宣言型プログラミング
	\item 構造化プログラミング
	\item メタプログラミング
	\item 論理プログラミング
	\item イベント駆動型プログラミング
	\item ドメイン固有言語
	\item スクリプト言語
	\item マクロ言語
	\item マークアップ言語
	\item HTML
	\item XML
	\item 動的型付けと静的型付け
	\item 動的スコープと静的スコープ
	\item 動的リンクと静的リンク
	\item ダックタイピング
	\item メモ化
	\item 自己書き換えコード
	\item シリアライズ
	\item 評価戦略
	\item 多重ディスパッチ (マルチメソッド)
	\item 糖衣構文
	\item ラムダ計算
	\item 表明 (アサーション)
	\item モナド
	\item ガベージコレクション
	\item マークアンドスイープ
	\item 参照カウント
	\item 名前修飾
	\item 弱い参照
	\item ランタイムライブラリ
	\item リファクタリング
	\item API
	\item 第一級オブジェクト
	\item バイトコード
	\item プラグイン
	\item スレッドセーフ
	\item リエントラント
	\item メインループ (イベントループ)
	\item 呼び出し規約
	\item バッファオーバーラン
	\item 副作用
	\item スパゲティプログラム
\end{itemize}
\end{multicols}

\section{言語理論}
\begin{multicols}{2}
\begin{itemize}
	\item チョムスキー階層
	\item バッカスナウア記法 (BNF)
	\item チョムスキー標準形
	\item グライバッハ標準形
	\item 形式文法と形式言語
	\item 字句解析
	\item lexとyacc
	\item 構文解析
	\item 曖昧な文法
	\item 抽象構文木
	\item LR構文解析
	\item クロスコンパイラ
	\item プッシュダウンオートマトン
	\item CYK法
	\item アーリー法
	\item マルコフアルゴリズム
	\item カリー化
	\item 機能的再帰関数 ($\mu$関数)
\end{itemize}
\end{multicols}

\clearpage

\section{データベース}
\begin{multicols}{2}
\begin{itemize}
 	\item 関係モデル
	\item データモデル
	\item E-Rモデル
	\item 関係演算
	\item 論理表現と物理表現
	\item データベース言語
	\item データベース管理システム
	\item 3層スキーマ
	\item トランザクション
	\item トランザクション分離レベル
	\item ACID特性
	\item ロールフォワードとロールバック
	\item 再編成と再構築
	\item ジャーナルファイル
	\item データウェアハウス
	\item 一元管理とその利点
	\item NULL値
	\item 更新時異常
\end{itemize}
\end{multicols}

\section{ソフトウェア開発}
\begin{multicols}{2}
\begin{itemize}
	\item アジャイルソフトウェア開発
	\item ソフトウェアインスペクション
	\item 形式仕様記述
	\item UML
	\item Rapid Application Development
	\item オープンソースソフトウェア
	\item クロスプラットフォーム
	\item 動的プログラム解析
	\item 統合開発環境
	\item フロントエンドとバックエンド
	\item モジュール
	\item リバースエンジニアリング
	\item 契約プログラミング
	\item アンチパターン
	\item デザインパターン
	\item 凝集度 (コヒージョン)
	\item 結合度
	\item 粒度
	\item 移植性
	\item ブラックボックスとホワイトボックス
\end{itemize}
\end{multicols}

\section{多変量解析と人工知能}
\begin{multicols}{2}
\begin{itemize}
	\item 傾向推定
	\item データクラスタリング
	\item エキスパートシステム
	\item データマイニング
	\item 因子分析
	\item 判別分析
	\item 正準相関分析
	\item オートエンコーダー
	\item ディープラーニング
	\item $k$近傍法
	\item 過剰適合
	\item 交差検証
	\item ブースティング
	\item ロジスティック回帰
	\item パーセプトロン
	\item シグモイド関数
	\item Q学習
	\item 自己組織化写像
	\item 集団的知性
	\item ベイズ誤り率
	\item 赤池情報量基準
	\item ブートストラップ法
	\item マハラノビス距離
	\item コーパス
	\item Neuroevolution
	\item ハイパーパラメータ
	\item 次元の呪い
	\item フレーム問題
	\item 人工無脳
	\item ソフトウェアエージェント
\end{itemize}
\end{multicols}

\clearpage

\section{データ構造とアルゴリズム}
\begin{multicols}{2}
\begin{itemize}
	\item 逆ポーランド記法とポーランド記法
	\item 優先度付きキュー
	\item ハッシュ関数
	\item チャイティンの定数 (停止確率)
	\item クラスPとクラスNP
	\item P $\neq$ NP予想
	\item コルモゴロフ複雑性
	\item 充足可能性問題
	\item 停止性問題
	\item ソートの安定性
	\item 内部ソートと外部ソート
	\item マトロイド
	\item 反復深化深さ優先探索
	\item 深さ制限探索
	\item 最良優先探索
	\item 双方向探索
	\item 均一コスト探索
	\item $\mathrm{A}^{*}$アルゴリズム
	\item BM法
	\item KMP法
	\item ノーフリーランチ定理
	\item 線形計画法
	\item 整数計画問題
	\item シンプレックス法
	\item Nelder-Mead法
	\item 最急降下法
	\item 共役勾配法
	\item 動的計画法
	\item 貪欲法
	\item ラグランジュの未定乗数法
	\item バックトラッキング
	\item ビタビアルゴリズム
	\item ヒューリスティクス
	\item 焼きなまし法
	\item タブーサーチ
	\item 粒子群最適化
	\item 蟻コロニー最適化
	\item 近似アルゴリズム
	\item 遺伝的アルゴリズム
	\item 最大フロー問題
	\item ギフト包装法
	\item ゲーム木
	\item ミニマックス法	
	\item $\alpha \beta$法
	\item ミラーラビン素数判定法
	\item テイラー展開
	\item 最小二乗法
	\item LU分解
	\item 線形合同法
	\item メルセンヌツイスタ
	\item メトロポリス法
	\item 特異値分解
\end{itemize}
\end{multicols}

\section{コンピュータグラフィクス}
\begin{multicols}{2}
\begin{itemize}
	\item フラクタル
	\item パーティクル
	\item メタボール
	\item レイトレーシング
	\item バンプマッピング
	\item 透視投影と平行投影
	\item ARとVR
	\item ブレゼンハムのアルゴリズム
	\item ダブルバッファ方式
	\item アフィン変換
	\item ワイヤフレームモデル
	\item サーフェスモデル
	\item CSG
	\item NURBS
	\item ポリゴン
	\item サブディビジョンサーフェス(細分割曲面)
	\item Zバッファ
	\item バイナリ空間分割
	\item 光モデル
	\item ラジオシティ
	\item マーチングキューブ法
	\item モーションブラー
	\item モーフィング
	\item GPU
\end{itemize}
\end{multicols}

\clearpage

\section{コンピュータアーキテクチャ}
\begin{multicols}{2}
\begin{itemize}
	\item マイクロアーキテクチャ
	\item バイナリ変換
	\item コード密度
	\item オープンアーキテクチャ
	\item ノイマン型コンピュータ
	\item ノイマンボトルネック
	\item DNAコンピュータ
	\item 量子コンピュータ
	\item 設定ファイル
	\item ハードウェア記述言語
	\item メタデータ
	\item ウェアレベリング
	\item リアルタイムシステム
	\item ウォッチドッグタイマ
	\item 透過性
	\item フォールトトレラント
	\item GUIとCUI
	\item ムーアの法則
	\item アムダールの法則
	\item グロシュの法則
	\item ベンチマーク
	\item RFID
	\item シフトレジスタ
	\item 抽象化
	\item ステータスレジスタ
	\item メモリマップトI/O
	\item I/OマップトI/O (ポートマップトI/O)
	\item DMA
	\item コントロールストア
	\item ビジーウェイト (ポーリング)
	\item 割込み
	\item 割込みハンドラ
	\item デバイスドライバ
	\item カーネル
	\item シェル
	\item システムコール
	\item プロセス制御ブロック
	\item CPUモードとユーザモード
	\item 実行ファイル
	\item コンテキストスイッチ
	\item マルチタスク
	\item プリエンプション
	\item ラウンドロビンスケジューリング
	\item 排他制御
	\item 競合状態
	\item デッドロック
	\item 食事する哲学者の問題
	\item リソーススタベーション
	\item クリティカルセクション
	\item 不可分操作
	\item テストアンドセット
	\item コンペアアンドスワップ
	\item デッカーのアルゴリズム
	\item 互換性
	\item ロック
	\item スピンロック
	\item セマフォ
	\item モニタ
	\item ページ方式とセグメント方式
	\item アドレス空間
	\item MMU
	\item ページテーブル
	\item ページフォールト
	\item セグメンテーション違反
	\item 仮想記憶
	\item ページ置換アルゴリズム
	\item LRU
	\item 参照の局所性
	\item 補助記憶装置
	\item ファイルシステム
	\item フラグメンテーション
	\item メインフレーム
	\item フレームバッファ
	\item スループット
	\item MIPS
	\item FLOPS
	\item 命令セット
	\item テーブルジャンプ
	\item プログラムカウンタ
	\item コールスタック
	\item コアダンプ
	\item アドレッシングモード
	\item インライン展開
	\item アキュムレータ
	\item 2進化10進数 (BCD)	
	\item 符号付き数値表現
	\item 可変長数値表現	
	\item 単精度と倍精度
	\item 丸め誤差
	\item 桁落ち誤差
	\item 算術シフトと論理シフト
	\item 非正規化数
	\item NaN
	\item 文字コード
	\item JISコード
	\item ASCII
	\item バレルシフタ
	\item ALU
	\item FPU
	\item ブースの乗算アルゴリズム
	\item キャリーセーブアダー
	\item 演算子強度低減
	\item SRT除算
	\item スティッキービット
	\item コプロセッサ
	\item レジスタファイル
	\item 例外処理
	\item VLIW
	\item データ依存
	\item ワイヤードロジック
	\item ソフトウェアパイプライニング
	\item ループ展開
	\item 投機的実行
	\item レベル適応型分岐予測
	\item 遅延スロット
	\item プリフェッチ
	\item リザベーションステーション
	\item メモリバリア
	\item レジスタリネーミング
	\item ファームウェア
	\item FPGA
	\item RAMとROM
	\item SRAMとDRAM
	\item SDRAM
	\item キャッシュミスの3C
	\item ダイレクトマップ方式
	\item セットアソシアティブ方式
	\item フルアソシアティブ方式
	\item スラッシング
	\item キャッシュコヒーレンシ
	\item ライトスルーとライトバック
	\item メモリインタリーブ
	\item L2キャッシュ (2次キャッシュ)
	\item プロセス識別子
	\item ノンブロッキングキャッシュ
	\item DIMM
	\item バックプレーン
	\item シンクロナイザ
	\item デファクトスタンダード
	\item SCSI
	\item PCI
	\item USB
	\item デイジーチェーン
	\item RAID
	\item ディスクアレイ
	\item 再構成可能コンピューティング
	\item 分散コンピューティング
	\item 分散共有記憶 (DSM)
	\item シンクライアント
	\item デスクトップ仮想化
	\item マルチプロセッサとマルチコア
	\item 対称型マルチプロセッシング
	\item ハードウェアマルチスレッディング
	\item NUMA
	\item コンピュータクラスタ
	\item メッセージ交換
	\item バリア同期
	\item MESIプロトコル
	\item 超並列マシン
	\item フリンの分類
	\item 性能解析 (プロファイラ)
	\item 仮想機械
	\item ハイパーバイザ
	\item エミュレータ
	\item 準仮想化
	\item GPGPU
\end{itemize}
\end{multicols}


\clearpage
\section{暗号理論とセキュリティ}
\begin{multicols}{2}
\begin{itemize}
	\item 安全素数
	\item Sボックス
	\item ステガノグラフィー
	\item 一方向性関数
	\item 暗号学的ハッシュ関数
	\item ゼロ知識証明
	\item 電子署名
	\item S/MIME
	\item セキュリティのCIA
	\item サンドボックス	
	\item 最小権限の原則
	\item ワンタイムパスワード
	\item 計算量的安全性
	\item 情報理論的安全性
	\item ケーパビリティ
	\item アクセス制御リスト
	\item ハードコア述語
	\item ワンタイムパッド
	\item 証明書失効リスト
	\item セキュリティホール
	\item 換字式暗号
	\item 転置式暗号
	\item DES
	\item ブロック暗号
	\item ストリーム暗号
	\item RSA暗号
	\item 差分解読法
	\item 線形解読法
	\item 頻度分析
	\item 締め上げ暗号分析
	\item 辞書攻撃
	\item DoS攻撃とDDoS攻撃
	\item 総当たり攻撃
	\item 誕生日攻撃
	\item 反射攻撃
	\item 暗号文単独攻撃
	\item ゼロデイ攻撃
	\item Fork爆弾
	\item SQLインジェクション
	\item DNSキャッシュポイゾニング
	\item サイドチャネル攻撃
	\item ゾンビコンピュータ
	\item コンピュータウイルス
	\item ブートストラップウイルス
	\item マクロウイルス
	\item ワーム
	\item スパムメール
	\item ボットによる攻撃
	\item トロイの木馬
	\item スパイウェア
	\item ファーミング
	\item ワンクリックウェア
	\item ソーシャルエンジニアリング
	\item スキミング
\end{itemize}
\end{multicols}

\section{メカトロニクス}
\begin{multicols}{2}
\begin{itemize}
	\item アクチュエータ
	\item ポテンショメータ
	\item ロータリエンコーダ
	\item サイリスタ
	\item サーミスタ
	\item マニピュレータ
	\item ステッピングモータ (パルスモータ)
	\item リーク電流
	\item チャタリング
	\item 能動回路と受動回路
	\item 分布定数回路
	\item 水晶振動子
	\item MEMS
	\item CMOS
	\item ECL
	\item LED
	\item PLL
	\item 分解能
	\item 測域センサ
	\item トレーサビリティ
\end{itemize}
\end{multicols}

\clearpage
\section{コンピュータネットワーク}
\begin{multicols}{3}
\begin{itemize}
	\item LAN, WAN, PAN
	\item Ethernet
	\item MIMO
	\item ARPANET
	\item OSI参照モデル
	\item エンドツーエンド原理
	\item オーバレイネットワーク
	\item 電力線搬送通信
	\item 回線交換
	\item 同期方式
	\item ガードインターバル
	\item 前方誤り訂正
	\item ADSL
	\item CSMA/CAとCSMA/CD
	\item ストリーミング
	\item 非同期転送モード
	\item フレームリレー
	\item FTTH
	\item 輻輳制御
	\item 自動再送要求
	\item 肯定要求 (ACK)
	\item VPN
	\item Ajax
	\item CIDR
	\item ネットワークアドレス変換
	\item クラウドコンピューティング
	\item プロキシサーバ
	\item リバースプロキシ
	\item イントラネット
	\item エクストラネット
	\item ルーティングプロトコル
	\item OSPF
	\item EIGRP
	\item RIP
	\item BGP
	\item モデム
	\item ISDN
	\item リピータ
	\item スイッチングハブ
	\item ネットワークカード (NIC)
	\item SIMカード
	\item TCP/IP
	\item 3ウェイハンドシェイク
	\item DHCP
	\item FTP
	\item HTTP
	\item IMAP
	\item IRC
	\item LDAP
	\item MGCP
	\item NNTP
	\item NTP
	\item POP
	\item RPC
	\item RTP
	\item SIP
	\item SMTP
	\item SNMP
	\item SSH
	\item Telnet
	\item TFTP
	\item SSL
	\item XMPP
	\item SCTP
	\item RSVP
	\item IPv4とIPv6
	\item IPフラグメンテーション
	\item ICMP
	\item IGMP
	\item IPSec
	\item VoIP
	\item Mobile IP
	\item RADIUS
	\item ARPとRARP
	\item トンネリング
	\item SPB
	\item PPP
	\item メディアアクセス制御
	\item ファイアウォール
	\item 非武装地帯 (DMZ)
	\item ホスト名
	\item ドメイン名とサブドメイン名
	\item サブネットマスク
	\item ラウンドトリップタイム
	\item ローミング
	\item ASP
	\item WWW
	\item URL
	\item ハイパーテキスト
	\item ハイパーリンク
	\item SNS
	\item コンテンツデリバリネットワーク
	\item VPS
	\item IaaS, PaaS, SaaS
	\item GPS
	\item GSM
	\item ACD
	\item Wi-Fi
	\item LTE
	\item WiMAX
	\item サービスセット識別子 (SSID)
\end{itemize}
\end{multicols}

\clearpage

\section{情報通信理論}
\begin{multicols}{2}
\begin{itemize}
	\item 情報量 (エントロピー)
	\item 結合エントロピー
	\item 相互情報量
	\item マルコフ過程
	\item レイテンシ
	\item SN比
	\item 符号誤り率
	\item スペクトル効率
	\item 通信路容量
	\item シャノン=ハートレーの定理
	\item 冗長性
	\item エンコードとデコード
	\item 巡回冗長検査
	\item 誤り訂正符号
	\item チェックサム
	\item パリティ検査符号
	\item ハフマン符号
	\item ハミング符号
	\item リードソロモン符号
	\item BCH符号
	\item 畳み込み符号
	\item ターボ符号
	\item 8B/10B
	\item 振幅変調
	\item 周波数変調
	\item 位相変調
	\item 位相偏移変調
	\item パルス符号変調
	\item 空間分割多重
	\item 周波数分割多重
	\item 時間多重
	\item 符号多重
	\item 直交周波数分割多重 (OFDM)
	\item 符号分割多重接続 (CDMA)
\end{itemize}
\end{multicols}

\section{信号画像処理}
\begin{multicols}{2}
\begin{itemize}
	\item ウェーブレット変換
	\item Z変換
	\item 双一次変換
	\item 高速フーリエ変換
	\item アダマール変換
	\item 無限インパルス応答
	\item スペクトル密度
	\item ハイパス/ローパスフィルタ
	\item 連長圧縮
	\item 窓関数
	\item ホワイトノイズとピンクノイズ
	\item 解像度
	\item ダイナミックレンジ
	\item ピーク信号対雑音比
	\item アスペクト比
	\item YUV色空間
	\item HSV色空間
	\item テンプレートマッチング
\end{itemize}
\end{multicols}

\section{制御工学}
\begin{multicols}{2}
\begin{itemize}
	\item ラウス・フルビッツの安定判別法
	\item ファジィ制御
	\item ボード線図
	\item ナイキスト線図
	\item ニコルズ線図
	\item 状態空間
	\item 可制御性と可観測性
	\item 有界入力有界出力安定性
	\item 状態観測器
	\item 時不変システム
	\item オンオフ制御
	\item ヒステリシス
	\item プログラマブルロジックコントローラ
	\item リアプノフ関数
\end{itemize}
\end{multicols}
\end{document}